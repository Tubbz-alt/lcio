\documentclass[11pt]{amsart}
\usepackage{geometry}                % See geometry.pdf to learn the layout options. There are lots.
\geometry{letterpaper}                   % ... or a4paper or a5paper or ... 
%\geometry{landscape}                % Activate for for rotated page geometry
%\usepackage[parfill]{parskip}    % Activate to begin paragraphs with an empty line rather than an indent
\usepackage{graphicx}
\usepackage{amssymb}
\usepackage{epstopdf}
\DeclareGraphicsRule{.tif}{png}{.png}{`convert #1 `dirname #1`/`basename #1 .tif`.png}

\title{LCIO: Large Scale Filesystem Aging}
\author{Authors}
%\date{}                                           % Activate to display a given date or no date

\begin{document}
\maketitle
%\section{}
%\subsection{}

\begin{abstract}
Judging how a filesystem's performance changes over time is not a trivial task. There have been
several different approaches in recent history. `Impressions' <cite> is a tool that statically
generates file system images, useful for seeing how metadata performance changes with respect to 
utilization. A newer approach is `Geriatrix' <cite>, which brings in a time component to the 
equation. Geriatrix queues operations to approach both a file size distribution and a relative 
time distribution. This takes into account the innate fragmentation of a filesystem induced by 
usage over time. 
\end{abstract}


\end{document}  